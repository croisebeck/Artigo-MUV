\documentclass[12pt]{article}

\usepackage{artigo-muv}

\usepackage{graphicx,url}

\usepackage[brazilian]{babel}
\usepackage[utf8]{inputenc}
\usepackage[T1]{fontenc}


%\usepackage[brazil]{babel}   
%\usepackage[latin1]{inputenc}


\sloppy

\title{MUV - Mais Um Voo, Simulador para Estimar Atrasos Conforme Fluxo de
  Passageiros no Aeroporto de Guarulhos}

\author{Miguel Antonio Copatti\inst{1},Roger Wesler Grabin\inst{2}, Anderson Seiji Ishiii\inst{3}}

\address{Escola do Mar, Ciência e Tecnologia\\
  Universidade do Vale do Itajaí (UNIVALI)\\
  Caixa Postal 360 -- 88302-202 -- Itajaí -- SC -- Brazil
%\nextinstitute
  %PEGAR E-MAIL DOS PARÇAS
  \email{\{miguel\_copatti,roger,anderson\}@live.com}
}

\begin{document} 

\maketitle

\begin{abstract}
  %TRADUZIR RESUMO
  This meta-paper describes the style to be used in articles and short papers
  for SBC conferences. For papers in English, you should add just an abstract
  while for the papers in Portuguese, we also ask for an abstract in
  Portuguese (``resumo''). In both cases, abstracts should not have more than
  10 lines and must be in the first page of the paper.
\end{abstract}
     
\begin{resumo} 

  %contexto do trabalho 1 paragrafo
  %objetivo geral 1 paragrafo
  %apresentar metodologia 1 a 2 paragrafos
  %apresentar principais resultados 1 a 2 paragrafos 

  este meta-artigo descreve o estilo a ser usado na confeco de artigos e
  resumos de artigos para publicao nos anais das conferncias organizadas
  pela SBC. solicitada a escrita de resumo e abstract apenas para os artigos
  escritos em portugus. Artigos em ingls devero apresentar apenas abstract.
  Nos dois casos, o autor deve tomar cuidado para que o resumo (e o abstract)
  no ultrapassem 10 linhas cada, sendo que ambos devem estar na primeira
  pgina do artigo.
\end{resumo}


\section{Introdução}

%Apresentar o contexto de aplicação: do que se trata e que outros
%lugares/universidades/pesquisas também encaram este problema (1 a 2
%parágrafos)
  Atuamente o aeroporto internacional de Guarulhos ostenta o título de maior
  aeroporto do Brasil. Sua capacidade total de embarque e de desembarque é
  de 15.352 considerando todos voos, domésticos e internacionais. A maior
  movimentação ocorre no terminal 3 (internacional) onde o fluxo  é de
  8.333 passageiro/hora. Contudo a capacidade de passageiro/hora pode ser 
  afetada pela quantidade de aeronaves estacionadas no pátio do terminal
  aeroportuário. Mesmo em tempo de recessão em 2017 o aeroporto registrou 
  aumento de 3,2\% se comparado com os dados de 2016. O Aeroporto planeja
  aumento no volume de passageiros e espera receber 60 milhões passageiro/ano,
  frente ao 36,6 milhões registrados no ano anterior. Para isso seja possível,
  será construído um novo pátio para aeronave e um novo pier até 2021 para o 
  terminal 3.

  Desde da sua inauguração no ano 1985, com uma área de 14 km quadrado,
  obteve uma movimentação de pessoas estimada em  37 milhões de passageiras
  com uma lotação total cerca de 40 milhões de aviões. Portanto observa-se
  que a capacidade máxima de passageiros será alcançada em um momento de
  pico, ocasionado por atrasos imprevistos ou datas comemorativas.
  
  % Fundamentar aspectos de Modelagem/Conceitos de Trabalhos similares e de
%estratégias utilizadas na elaboração do trabalho



%Apresentar o objetivo geral do artigo;

%verificar esse trexo
Este artigo é estruturado da seguinte maneira: a Seção 2 apresenta alguns trabalhos
relacionados a este, apontando algumas das abordagens discutidas na literatura,
em seguida, a Seção 3 apresenta a metodologia e técnicas utilizadas para a construção 
do simulador e seus artefatos produzidos neste trabalho, a Seção 4 descreve e analisa 
os resultados obtidos através da utilização do simulador, por fim, a seção 5 e última 
discute sobre algumas conclusões extraídas a partir dos resultados coletados e da 
análise do simulador, objetivos atingidos e melhorias para o simulador.
  

\section{Revisão bibliográfica} \label{sec:revisaobibliografica}

A diferença crescente entre a demanda prevista e o número de
operações efetivamente realizadas impõe aos usuários
restrições de oferta de voos pelas companhias aéreas que,
impedidas de aumentar suas frequências em Congonhas, são 
obrigadas a criar novos voos partindo de outros aeroportos,
como Guarulhos e Campinas \cite{boulic:91}.

Em contrapartida, a ANAC vem adotando medidas de restrição 
de tráfego cada vez mais severas em Congonhas, tais como: 
alocação de slots para operações, proibição de operação de 
aeronaves comerciais na pista auxiliar e determinação de tempo
máximo de permanência de aeronaves nos boxes de estacionamento,
entre outras. 


\section{Metodologia}

Utilizando da simulação discreta para modelar o problema afim de estimar
quando ocorre maior demanda por voos no aeroporto de Guarulhos, seja por
fatores externos, como alta demanda de passageiros, fatores climaticos,
objetos na pista, entre outros tipos de eventos que podem gerar algum 
tipo de atraso no aeroporto.


\section{Configuração do aeroporto}

A configuração do sistemas de pistas  em geral , o fator mais importante
para determinar a capacidade de um aeroporto , sendo que o mais comum 
gargalo do sistema aeroportuário como um todo . Quando capacidade do 
sistema de pistas é excedido , o aeroporto invariavelmente começa a 
sofrer atrasos.

Atualmente o aeroporto de Guarulhos possui 2 pistas , a primeira foi 
inaugurada em 1985 com  3000 metros , a segunda possui 3500 metros, 
mais tarde ampliada para 3700 metros , há ainda previsto uma terceira
pista com extensão de 1500 metros, porém ainda não foi construída. 

\subsection{Subsections}

The subsection titles must be in boldface, 12pt, flush left.

\section{Detalhes do Simulador}\label{sec:simulador}


Figure and table captions should be centered if less than one line
(Figure~\ref{fig:exampleFig1}), otherwise justified and indented by 0.8cm on
both margins, as shown in Figure~\ref{fig:exampleFig2}. The caption font must
be Helvetica, 10 point, boldface, with 6 points of space before and after each
caption.

\begin{figure}[ht]
\centering
\includegraphics[width=.5\textwidth]{fig1.jpg}
\caption{A typical figure}
\label{fig:exampleFig1}
\end{figure}

\begin{figure}[ht]
\centering
\includegraphics[width=.3\textwidth]{fig2.jpg}
\caption{This figure is an example of a figure caption taking more than one
  line and justified considering margins mentioned in Section~\ref{sec:figs}.}
\label{fig:exampleFig2}
\end{figure}

In tables, try to avoid the use of colored or shaded backgrounds, and avoid
thick, doubled, or unnecessary framing lines. When reporting empirical data,
do not use more decimal digits than warranted by their precision and
reproducibility. Table caption must be placed before the table (see Table 1)
and the font used must also be Helvetica, 10 point, boldface, with 6 points of
space before and after each caption.

\begin{table}[ht]
\centering
\caption{Variables to be considered on the evaluation of interaction
  techniques}
\label{tab:exTable1}
\includegraphics[width=.7\textwidth]{table.jpg}
\end{table}

\section{Images}

All images and illustrations should be in black-and-white, or gray tones,
excepting for the papers that will be electronically available (on CD-ROMs,
internet, etc.). The image resolution on paper should be about 600 dpi for
black-and-white images, and 150-300 dpi for grayscale images.  Do not include
images with excessive resolution, as they may take hours to print, without any
visible difference in the result. 

\section{References}

Bibliographic references must be unambiguous and uniform.  We recommend giving
the author names references in brackets, e.g \cite{boulic:91}.

The references must be listed using 12 point font size, with 6 points of space
before each reference. The first line of each reference should not be
indented, while the subsequent should be indented by 0.5 cm.

\bibliographystyle{sbc}
\bibliography{artigo-muv}

\end{document}
